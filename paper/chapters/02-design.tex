\chapter{Design and materials}
In this section we will discuss the design of our mechanical, electronic, and software systems and the materials used.

\section{Mechanical design}
The mechanical design consists of three main components: horizontal movement, vertical movement, and rotating the racket. 
The materials utilized are a combination of purchased components such as linear rails, timing belts, bolts, and nuts 
alongside custom 9MM MDF parts cut using a laser cutter in FabLab.

Two 1.5-meter-long linear rails are positioned parallel to each other along the table's edge to ensure stability and prevent tipping of the system. 
Three linear rail blocks, which slide along these rails, support the vertical structure of the design. 
These rail blocks have screw holes, allowing them to be securely connected in a triangular formation using an MDF plate.
The vertical structure is mounted on this plate.

A stepper motor is mounted at one corner of the table, fixed to the edge of the rails using MDF. 
This motor drives a 60-tooth GT2 gear. 
At the opposite corner, a shoulder bolt with a gear is similarly secured using MDF. 
A 6mm-wide GT2 timing belt spans between these two gears, and is attachted to one of the horizontal rail blocks, allowing horizontal movement by rotating the stepper motor.

The vertical structure is mounted on the horizontal MDF plate, attachted to the horizontal linear blocks. 
It includes a 0.5-meter-long aluminium pipe and a vertical MDF plate that stabilizes the pipe.
A gear is placed on top of the vertical plate and pipe, held in place with a shoulder bolt.
A vertical rail block is positioned on the pipe, allowing vertical movement.
Another stepper motor is attached at the bottom of the horizontal MDF plate, with a gear connected to it.
A timing belt spans between these two gears and is attached to the vertical rail block, allowing vertical movement by rotating the stepper motor.

To rotate the racket, a servo motor is attached to the vertical rail block using MDF. 
A solenoid is mounted on the rotating components of the servo motor, and the racket is attached to a small MDF piece connected to the solenoid.
 
This mechanical design enables the racket to move left and right, up and down,
rotate 90 degrees around its axis, and push the ping-pong ball.

\begin{figure}[h] 
	\centering \includegraphics[height=3cm]{./images/frontrender.jpg}
	\caption{A render of the mechanical design from the front side.}
\end{figure}
\begin{figure}[h] 
	\centering \includegraphics[height=3cm]{./images/backrender.jpg}
	\caption{A render of the mechanical design from the back side.}
\end{figure}



\section{Block Diagram}
Below is a block diagram showing the general control flow of the entire system and all its components.

\begin{figure}[h]
	\centering\includegraphics[height=3cm]{./images/blockdiagram}
	\caption{System Design Block Diagram}
\end{figure}

\section{Electronic Components and Controllers}
The system mainly relies on 2 mobile phones which are used for collecting the position information of the ping pong ball with 3 degrees of freedom. The mobile phones are configured with an app to act as IP cameras, which are then accessed over Wi-Fi on the control device, which is a laptop running our software.

The ESP32 microcontroller is connected to 3 stepper motor driver modules (DRV8825), which are in turn connected to 3 R.T.A. bipolar stepper motors which control the gantry. The ESP32 acts as a master to a PIC18F, though a connected radio frequency communication module (NRF24), which is used to transmit data to another NRF24 module, connected to the PIC18F slave microcontroller. This data consists of information on when to "fire" the racket to hit the ball, by activating the optcoupled 12v solenoid, and information on what angle the racket should be at, by controlling the 6v servo motor controlling the racket.

\begin{figure}[h]
	\centering\includegraphics[height=3cm]{./images/steppermotor}
	\caption{Stepper Motor Used}
\end{figure}

\section{Software Design}
The main software which controls the system is ran a separate computer, due to the speed constraints of small microcontrollers. The code connects to the IP camera phones, reads the streamed video data, and process it using ball detection algorithms to determine the position of the ping pong ball in 3D space. The software sends a requested gantry position, a requested racket orientation, and when the racket should "fire" over Wi-Fi to the ESP32 microcontroller, which then sends this data over radio using the NRF24 to the PIC18F.

The software additionally maintains the game state through a simplified ruleset, allowing the score and current ball position to be visualized in real time. The software determines the ball's position and additionally applies a predictive model to determine where the gantry should move to.


\section{Printed Circuit Boards}

Our design additionally consists of 2 Printed Circuit Boards (PCBs), one for each used microcontroller. The first PCB designed is for the PIC18F slave microcontroller. It contains connections for the solenoid, and its control circuit which consists of an optocoupler and a Darlington NPN bipolar transistor (TIP120). It also cor ontains connections for the servo motor and its control circuit which is a simple MOSFET transistor circuit. It also contains connections for the NRF24 module.

\begin{figure}[h]
	\centering\includegraphics[height=3cm]{./images/picpcb}
	\caption{PIC18F PCB Design}
\end{figure}

The second designed PCB is for the ESP32 master controller. This PCB consists of headers for the ESP32 controller itself, and headers for each of the DRV8825 modules. It has a switch for on board adjustment of the drivers' control pins. It also contains connections for the NRF24 module.

\begin{figure}[h]
	\centering\includegraphics[height=3cm]{./images/esppcb}
	\caption{ESP32 PCB Design}
\end{figure}

\section{Bill of Materials}

\begin{center}
\begin{tabular}{|l|c|c|c|}
 \hline
Item & Amount & Price (\texteuro) \\
 \hline\hline
 MDF (60x30) & 1 & 4 \\ 
 \hline
 5meter 6mm Timing Belt & 1 & 6.62\\
 \hline
 GT2 60-tooth gear & 4 & 3.19\\
 \hline
 0.5meter M16 Aluminium pipe & 1 & 5.9\\
 \hline
 1.5meter SBR16 Linear rail & 2 & 20.46\\
 \hline
 D5 M4 50mm Shoulder bolt & 2 & 2.94\\
 \hline
 Servo Motor & 1 & 3.5 \\
 \hline
 Solenoid & 1 & 6.9 \\ 
 \hline
 ESP32 & 1 & 6.9 \\
 \hline
 PIC18F & 1 & 2.4 \\
 \hline
 NRF24 & 2 & 6.68 \\
 \hline
 Stepper Motor & 3 & - \\
 \hline
 Stepper Driver DRV8825 & 3 & 13.5 \\
 \hline
 Optocoupler & 1 & - \\
 \hline
 Transistors & - & - \\
 \hline
 Resistors & - & - \\
 \hline
 Capacitors & - & - \\
 \hline
\end{tabular}
\end{center}